% It is an example file showing how to use the 'sigkddExp.cls' 
% LaTeX2e document class file for submissions to sigkdd explorations.
% It is an example which *does* use the .bib file (from which the .bbl file
% is produced).
% REMEMBER HOWEVER: After having produced the .bbl file,
% and prior to final submission,
% you need to 'insert'  your .bbl file into your source .tex file so as to provide
% ONE 'self-contained' source file.
%
% Questions regarding SIGS should be sent to
% Adrienne Griscti ---> griscti@acm.org
%
% Questions/suggestions regarding the guidelines, .tex and .cls files, etc. to
% Gerald Murray ---> murray@acm.org
%

\documentclass{sigkddExp}

\begin{document}
%
% --- Author Metadata here ---
% -- Can be completely blank or contain 'commented' information like this...
%\conferenceinfo{WOODSTOCK}{'97 El Paso, Texas USA} % If you happen to know the conference location etc.
%\CopyrightYear{2001} % Allows a non-default  copyright year  to be 'entered' - IF NEED BE.
%\crdata{0-12345-67-8/90/01}  % Allows non-default copyright data to be 'entered' - IF NEED BE.
% --- End of author Metadata ---

\title{6.864, Fall 2017 - Neural Approaches to Question Retrieval}
\subtitle{Codebase: https://github.com/shashank-srikant/6.864\_term\_project}

% You need the command \numberofauthors to handle the "boxing"
% and alignment of the authors under the title, and to add
% a section for authors number 4 through n.
%
% Up to the first three authors are aligned under the title;
% use the \alignauthor commands below to handle those names
% and affiliations. Add names, affiliations, addresses for
% additional authors as the argument to \additionalauthors;
% these will be set for you without further effort on your
% part as the last section in the body of your article BEFORE
% References or any Appendices.

\numberofauthors{2}
%
% You can go ahead and credit authors number 4+ here;
% their names will appear in a section called
% "Additional Authors" just before the Appendices
% (if there are any) or Bibliography (if there
% aren't)

% Put no more than the first THREE authors in the \author command
%%You are free to format the authors in alternate ways if you have more 
%%than three authors.

\author{
%
% The command \alignauthor (no curly braces needed) should
% precede each author name, affiliation/snail-mail address and
% e-mail address. Additionally, tag each line of
% affiliation/address with \affaddr, and tag the
%% e-mail address with \email.
\alignauthor Vadim Smolyakov*\\
       \affaddr{CSAIL, MIT}\\
       \email{vss@csail.mit.edu}
\alignauthor Shashank Srikant\titlenote{Author order decided by tossing a fair coin.}\\
       \affaddr{CSAIL, MIT}\\
       \email{shash@mit.edu}
}

\balancecolumns
\date{30 July 1999}
\maketitle
\begin{abstract}
Content on the internet grows at an exponential rate. Given this growth, finding relevant information accurately becomes a critical task for the NLP community to address. More so, with this rapid growth, curating labeled datasets to build models for the wide variety of content available on the internet has become extremely time and resource intensive. In this work, we explore whether neural models are able to successfully model content similarity tasks, and whether they can transfer knowledge from one domain, where supervised labels are available, to a domain with no available labels. Specifically, we explore content similarity in online discussion forums, where we explore the following questions - a. how effectively can neural approaches model question-answer similarity tasks i.e. given a question and answer pair present on an online discussion forum, how effectively can neural approaches find similar pairs on that forum. b. Given a model of question-answer similarity in one online discussion forum, how effective are neural approaches in transferring that knowledge to a new, loosely related domain. In this work, we explore a baseline approach of modeling question similarity tasks on the popular online community  \textit{AskUbuntu}. We show how neural architectures like LSTMs and CNNs compare to a traditional approaches in information retrieval. Additionally, and importantly, we explore the problem of transfering these models to detect question-answer similarity on \textit{Android stack exchange},  a similar yet different online discussion community which discusses Android related problems. We show how neural domain adaptation techniques successfully beat baseline IR techniques direct neural transfer techniques. We also discuss some limitations and challenges in using such architectures.
\end{abstract}

\section{Introduction}
The problem of text similarity, and specifically, similarity of short queries or answers found on the internet, have been central to the modern NLP community. With the explosion in content on the internet, a lack of robust tools to find similar content has the risk of creating further similar and redundant content, which only exacerbates the original problem.

Another relevant and pressing concern which such an explosion and variety of content has introduced is the increased cost of building predictive NLP, NLU models which cater to such content. The variety in content requires building a model from scratch, irrespective of how closely related the content may have been to a previously built model. For instance, if we were to model the reviews written for movies, we would have to reinvest effort and time in modeling reviews written for another domain, say, food or hotels. In spite of there being conceptual, semantic similarities between the tasks of reviewing movies and reviewing food, models would have to be created anew. And each such modeling exercise demands a large repository of curated, preferably labeled data, which at times is simply unfeasible. The pressing challenge such a variety of content has created is to be able to learn with minimal supervision, and from loosely related datasets.

In this work, we investigate two problems - one, to model question-similarity tasks using state of the art techniques in neural modeling and two, transfer those models to a loosely related, yet equally rich domain under the constraint of having no supervised information on that new domain. Specifically, we explore 

With recent advances in neural architectures for NLP applications, it is relevant to investigate how they perform on the task of question similarity. Traditionally, text similarity has been approached with standard information retrieval techniques, like Lucene and REF, which consider XYZ. It is relevant to investigate how more recent techniques like LSTMs and CNNs, which have shown to successfully model text on tasks like XYZ, model this particular task. 

In comparison to this, it is interesting to investigate whether using neural approaches, one can outperform them while not having to invest in engineering the right features to get to such a performance. In addition to 

\section{Related work}
Typically, the body of a paper is organized
into a hierarchical structure, with numbered or unnumbered
headings for sections, subsections, sub-subsections, and even
smaller sections.  The command \texttt{{\char'134}section} that
precedes this paragraph is part of such a
hierarchy.\footnote{This is the second footnote.  It
starts a series of three footnotes that add nothing
informational, but just give an idea of how footnotes work
and look. It is a wordy one, just so you see
how a longish one plays out.} \LaTeX\ handles the numbering
and placement of these headings for you, when you use
the appropriate heading commands around the titles
of the headings.  If you want a sub-subsection or
smaller part to be unnumbered in your output, simply append an
asterisk to the command name.  Examples of both
numbered and unnumbered headings will appear throughout the
balance of this sample document.

Because the entire article is contained in
the \textbf{document} environment, you can indicate the
start of a new paragraph with a blank line in your
input file; that is why this sentence forms a separate paragraph.

\section{In-domain Question similarity}
\subsection{Type Changes and Special Characters}
We have already seen several typeface changes in this sample.  You
can indicate italicized words or phrases in your text with
the command \texttt{{\char'134}textit}; emboldening with the
command \texttt{{\char'134}textbf}
and typewriter-style (for instance, for computer code) with
\texttt{{\char'134}texttt}.  But remember, you do not
have to indicate typestyle changes when such changes are
part of the \textit{structural} elements of your
article; for instance, the heading of this subsection will
be in a sans serif\footnote{A third footnote, here.
Let's make this a rather short one to
see how it looks.} typeface, but that is handled by the
document class file. Take care with the use
of\footnote{A fourth, and last, footnote.}
the curly braces in typeface changes; they mark
the beginning and end of
the text that is to be in the different typeface.

Citations to articles \cite{bowman:reasoning, clark:pct, braams:babel, herlihy:methodology},
conference
proceedings \cite{clark:pct} or books \cite{salas:calculus, Lamport:LaTeX} listed
in the Bibliography section of your
article will occur throughout the text of your article.
You should use BibTeX to automatically produce this bibliography;
you simply need to insert one of several citation commands with
a key of the item cited in the proper location in
the \texttt{.tex} file \cite{Lamport:LaTeX}.
The key is a short reference you invent to uniquely
identify each work; in this sample document, the key is
the first author's surname and a
word from the title.  This identifying key is included
with each item in the \texttt{.bib} file for your article.

The details of the construction of the \texttt{.bib} file
are beyond the scope of this sample document, but more
information can be found in the \textit{Author's Guide},
and exhaustive details in the \textit{\LaTeX\ User's
Guide}\cite{singh2016question}.

So far, this article has shown only the plainest form
of the citation command, using \texttt{{\char'134}cite}.
%
%You can also use a citation as a noun in a sentence, as
% is done here, and in the \citeN{herlihy:methodology} article;
% use \texttt{{\char'134}citeN} in this case.  You can
% even say, ``As was shown in \citeyearNP{bowman:reasoning}. . .''
% or ``. . . which agrees with \citeANP{braams:babel}...'',
% where the text shows only the year or only the author
% component of the citation; use \texttt{{\char'134}citeyearNP}
% or \texttt{{\char'134}citeANP}, respectively,
% for these.  Most of the various citation commands may
% reference more than one work \cite{herlihy:methodology,bowman:reasoning}.
% A complete list of all citation commands available is
% given in the \textit{Author's Guide}.
\section{Domain Adaptation}

\section{Other techniques for domain adaptation}

\section{Experiments}

\section{Discussion}

%ACKNOWLEDGEMENTS are optional
\section{Acknowledgements}
This section is optional; it is a location for you
to acknowledge grants, funding, editing assistance and
what have you.  In the present case, for example, the
authors would like to thank Gerald Murray of ACM for
his help in codifying this \textit{Author's Guide}
and the \textbf{.cls} and \textbf{.tex} files that it describes.

%
% The following two commands are all you need in the
% initial runs of your .tex file to
% produce the bibliography for the citations in your paper.
\bibliographystyle{abbrv}
\bibliography{sigproc}  % sigproc.bib is the name of the Bibliography in this case
% You must have a proper ".bib" file
%  and remember to run:
% latex bibtex latex latex
% to resolve all references
%
% ACM needs 'a single self-contained file'!
%
%APPENDICES are optional
% SIGKDD: balancing columns messes up the footers: Sunita Sarawagi, Jan 2000.
% \balancecolumns

% That's all folks!
\end{document}
